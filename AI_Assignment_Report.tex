\documentclass{article}

\usepackage{graphicx}
\usepackage{algorithm}
\usepackage{algpseudocode}
\usepackage{authblk}
\usepackage{url}
\usepackage[utf8]{inputenc}
\usepackage{color}
\usepackage{amsmath}
\graphicspath{{C:\Users\tkhal\Desktop\College\Second Yr\Data Analysis EE201\Project\Question 2}}

\title{AI Assignment Report}
\vspace{20pt}
\vspace{20pt}
\date{\today}
\author{Tabish Khalid Halim , \\ 200020049 \\ Anand Hegde , \\ 200020007}
\vspace{20pt}
\affil{Department of Computer Science, IIT Dharwad}
\begin{document}

\maketitle
\pagenumbering{gobble}
\newpage
\tableofcontents

\newpage
\pagenumbering{arabic}
\section*{Problem Statement :}
\textbf{(25 points)It’s still about coin toss:}You have been given the data (50000 samples) of arandom  variableZ.   You  know  that $Z=X+ 10Y$,  where X is a uniform  random  variable between -3 and 3.  You also know that 
$$ Y = \sum_{i = 1}^{k} W_i $$
where $ k \in {2,3,4}$ , $W_i$ are independent and identically distributed (i.i.d.)  and belong to oneof the following:
\begin{itemize}
    \item Exponential distribution characterized by its mean $ \frac{1}{\lambda} $
    \item Rayleigh distribution characterized by $\sigma$ \\ (\url{https://en.wikipedia.org/wiki/Rayleigh_distribution})
    \item Half-normal distribution characterized by $\sigma$ \\ (\url{https://en.wikipedia.org/wiki/Half-normal_distribution})
\end{itemize}    
Come up with a mechanism to find $k$ and the distribution of $W_i$ along with the characterizing parameter \textbf{rounded to the nearest integer} (mean if it is an exponential distribution and $\sigma$ otherwise).  Justify your mechanism analytically.
\vspace{20pt}
\section{Approach}
Well , the basic approach I used is to use mathematical calculations to find a value of k for each case \{ $W_i$ is Exponential Random Variable ; $W_i$ is Rayleigh's Random Variable ; $W_i$ is Half-normal Random Variable \} .
\vspace{10pt}
\\
Further after getting the value of k for each case , I found the rounded off the value of that k to the nearest integer and checked whether $k \in \{2,3,4\} $.
\vspace{10pt}
\\If $k \in \{ 2,3,4\}  $ , then the {\color{cyan} test case} is {\color{green}successful} , else the {\color{cyan} test case} is {\color{red}unsuccessful}.
\vspace{10pt}
\\In case of when there are multiple {\color{green}successful} test cases , I will be considering the test case having the least calculation error of k 
in my final answer for the probability distribution function type for $W_i$ .
\vspace{10pt}
\\After getting this test case , I can easily get it's characteristics through mathematical computation .
\newpage
\section{Variables used in Python Program}
\vspace{20pt}
\begin{itemize}
    \item \textbf{value} : Stores the count of the number of tails and head in the final data.
    \item \textbf{dataset} : The raw data received from .xls data file.
    \item \textbf{Z} : Stores raw data in numpy array
    \item \textbf{Mean} : Stores the mean of the dataset.
    \item \textbf{VAR} : Stores the variance of the dataset.
    \item \textbf{a} : Variable for handling the lower bound of X .
    \item \textbf{b} :  Variable for handling the upper bound of X .
    \item \textbf{k} : It stores the value of k which was mentioned in the question .
    \item \textbf{lambda} : It stores the characteristic feature for the probability distribution function in the case of Exponential distribution function.
    \item \textbf{sigma} : It stores the characteristic feature for the probability distribution function in the case of Rayleigh distribution function or Half-normal distribution function.
    \item \textbf{choice} : This stores the probability of the biasing of coin towards heads .
    \item \textbf{a} : This stores the choice of user whether he wants to print the final answer or not .
    \item \textbf{kdiff1,kdiff2,kdiff3} : These variables store the differences in calculation of k with its nearest integer value.
    \item \textbf{val1,val2,val3} : These variables store whether the rounded off value of k is within the set $k \in \{ 2,3,4\}  $ .
    \item \textbf{Arr} : This is the variable that stores calculation error in k if the test case is Successful .
    \item \textbf{exp\_dist} : Distribution Function variable for Exponential distribution function case.
    \item \textbf{ray\_dist} : Distribution Function variable for Rayleigh distribution function case.
    \item \textbf{half\_dist} : Distribution Function variable for Half-Normal distribution function case.
\end{itemize}
\newpage
\section{Functions created in Python Program}
\vspace{20pt}
\begin{itemize}
    \item \textbf{check()} : function to check whether each test case is true or false .
    \item \textbf{roundoff()} : function to round off the parameter passed to the nearest integer value .
    \item \textbf{roundoff1()} : function to round off the parameter passed to one digit decimal value .
    \item \textbf{exponential()} : function to run the case of exponential distribution function .
    \item \textbf{rayleigh()} : function to run the case of rayleigh distribution function .
    \item \textbf{halfnormal()} : function to run the case of half normal distribution function .
    \item \textbf{border()} :  function to print border style .
    \item \textbf{answer()} : function that returns the minimum value inside a list .
    \item \textbf{plot\_exp()} : function that plots the exponential distribution function .
    \item \textbf{plot\_rayleigh()} : function that plots the rayleigh distribution function .
    \item \textbf{plot\_halfnormal()} : function that plots the half-normal distribution function .
\end{itemize}
\newpage
\section{Logic used/Mathematical Computation}
The main logic for the code is to find the value of k while determining the type of distribution for $W_i$.
\vspace{5pt}
\\We can deduce from the question that X and Y are independent as well as each $W_i$ are independent .
\vspace{10pt}
\vspace{10pt}
\\We know that :
\begin{equation} \label{equation1}
    Z = X + 10Y 
\end{equation}
\\Also ,
\begin{equation} \label{equation2}
    Y = \sum_{i = 1}^{k} W_i
\end{equation}
\\Given , that X is a uniform  random  variable between -3 and 3.
\\So,
$$ a = -3 , b = 3$$
$$ E(X) = \frac{(b+a)}{2} = 0$$
$$ Var(X) = \frac{(b-a)^2}{12} = 3$$
\\Now we take Mean and Variance respectively on both sides in eqn \ref{equation1} and eqn \ref{equation2}.
\vspace{5pt}
\\So , we get :- 
\begin{equation}
    E(Z) = E(X) + 10E(Y)
\Rightarrow E(Y) = \frac{E(Z) - E(X)}{10}
\end{equation}
\vspace{5pt}
\begin{equation}
    Var(Z) = Var(X) + 100Var(Y)
\Rightarrow Var(Y) = \frac{Var(Z) - Var(X)}{100}
\end{equation}
\newpage
Now , we just go through each case for the distribution function type for $W_i$:
    \subsection{Exponential Distribution Function :}
    \vspace{5pt}
    \setcounter{equation}{0}
    \begin{equation}
        E(W_i) = \frac{1}{\lambda} 
    \end{equation}
    \begin{equation}
        Var(W_i) = \frac{1}{\lambda^2}
    \end{equation}
    \\So ,
    \begin{equation}
        E(Y) = \sum_{i = 1}^{k} E(W_i)
        \Rightarrow E(Y) = \frac{k}{\lambda}
    \end{equation}
    \begin{equation}
        Var(Y) = \sum_{i = 1}^{k} Var(W_i)
        \Rightarrow Var(Y) = \frac{k}{\lambda^2}
    \end{equation}
    \vspace{5pt}
    \\So , equating the eqns we get : 
    \begin{equation}
        E(Y) = \frac{k}{\lambda}  = \frac{E(Z) - E(X)}{10}
    \end{equation}
    \begin{equation}
        Var(Y) = \frac{k}{\lambda^2} = \frac{Var(Z) - Var(X)}{100}
    \end{equation}
    \vspace{5pt}
    \\Now , equating eqn 5 and eqn 6 we get the value of k as : 
    \begin{equation}
        k = \frac{(E(Z) - E(X))^2}{Var(Z) - Var(X)}
    \end{equation}
    \vspace{5pt}
    \\ and $\lambda$ is :
    \begin{equation}
        \lambda = \frac{10k}{E(Z) - E(X)}
    \end{equation}

    \subsection{Rayleigh Distribution Function :}
    \vspace{5pt}
    \setcounter{equation}{0}
    \begin{equation}
        E(W_i) = \sigma \sqrt{\frac{\pi}{2}} 
    \end{equation}
    \begin{equation}
        Var(W_i) = \left(\frac{4-\pi}{2}\right)\sigma^2 
    \end{equation}
    \\So ,
    \begin{equation}
        E(Y) = \sum_{i = 1}^{k} E(W_i)
        \Rightarrow E(Y) = k\sigma\sqrt{\frac{\pi}{2}} 
    \end{equation}
    \begin{equation}
        Var(Y) = \sum_{i = 1}^{k} Var(W_i)
        \Rightarrow Var(Y) = k\sigma^2\left(\frac{4-\pi}{2}\right)
    \end{equation}
    \vspace{5pt}
    \\So , equating the eqns we get : 
    \begin{equation}
        E(Y) = k\sigma\sqrt{\frac{\pi}{2}} = \frac{E(Z) - E(X)}{10}
    \end{equation}
    \begin{equation}
        Var(Y) = k\sigma^2\left(\frac{4-\pi}{2}\right) = \frac{Var(Z) - Var(X)}{100}
    \end{equation}
    \vspace{5pt}
    \\Now , equating eqn 5 and eqn 6 we get the value of k as : 
    \begin{equation}
        k = \left(\frac{(E(Z) - E(X))^2}{Var(Z) - Var(X)}\right)\left(\frac{4-\pi}{\pi}\right)
    \end{equation}
    \vspace{5pt}
    \\ and $\sigma$ is :
    \begin{equation}
        \sigma = \left(\frac{E(Z) - E(X)}{10k}\right)\left(\sqrt{\frac{2}{\pi}}\right)
    \end{equation}

    \subsection{Half-normal Distribution Function :}
    \vspace{5pt}
    \setcounter{equation}{0}
    \begin{equation}
        E(W_i) = \sigma \sqrt{\frac{2}{\pi}} 
    \end{equation}
    \begin{equation}
        Var(W_i) = \left(\frac{\pi-2}{\pi}\right)\sigma^2 
    \end{equation}
    \\So ,
    \begin{equation}
        E(Y) = \sum_{i = 1}^{k} E(W_i)
        \Rightarrow E(Y) = k\sigma \sqrt{\frac{2}{\pi}} 
    \end{equation}
    \begin{equation}
        Var(Y) = \sum_{i = 1}^{k} Var(W_i)
        \Rightarrow Var(Y) = \left(\frac{\pi-2}{\pi}\right)k\sigma^2
    \end{equation}
    \vspace{5pt}
    \\So , equating the eqns we get : 
    \begin{equation}
        E(Y) = k\sigma\sqrt{\frac{\pi}{2}} = \frac{E(Z) - E(X)}{10}
    \end{equation}
    \begin{equation}
        Var(Y) = \left(\frac{\pi-2}{\pi}\right)k\sigma^2 = \frac{Var(Z) - Var(X)}{100}
    \end{equation}
    \vspace{5pt}
    \\Now , equating eqn 5 and eqn 6 we get the value of k as : 
    \begin{equation}
        k = \left(\frac{(E(Z) - E(X))^2}{Var(Z) - Var(X)}\right)\left(\frac{\pi-2}{\pi}\right)\left(\frac{\pi}{2}\right)
    \end{equation}
    \vspace{5pt}
    \\ and $\sigma$ is :
    \begin{equation}
        \sigma = \left(\frac{E(Z) - E(X)}{10k}\right)\left(\sqrt{\frac{\pi}{2}}\right)
    \end{equation}
\section*{Logic :}
Now , we have gone through each case of probable probability distribution functions for $W_i$ ,
\vspace{5pt}
\\Next , we must first round of the value of k obtained and check whether the rounded off integer value is $\in \{2,3,4\}$ . 
\vspace{5pt}
\\If the value of $ k\in \{2,3,4\}$ then , the test case is labelled {\color{green}successful} but if not the it is labelled {\color{red}unsuccessful}.
\vspace{5pt}
\\If multiple test cases are found successful , then the test case which has the least deviation of the value of k from the nearest integer is taken as the true test case
and is considered for the final answer .
\newpage
\section{Graphical Analysis} 
\vspace{10pt}
\includegraphics[scale = 0.29]{Figure 1.jpg}
\vspace{5pt}
\\I have made the P.D.F(Probability Distribution Function) of $Y$ from the final P.D.F type obtained for $W_i$ .
\vspace{5pt}
\\Based on my dataset , I have gotten my P.D.F type for $W_i$ and $Y$ as \textbf{half-normal distribution function} .
\vspace{20pt}
\newpage
\section{Results}
\vspace{20pt}
The following conclusions have been made after evaluation of my program : 
\begin{enumerate}
    \item The P.D.F of $W_i$ is \textbf{half-normal distribution function} .
    \item The value of k is $2$.
    \item The value of $\sigma$ is $2$
\end{enumerate}
\vspace{20pt}
\subsection*{Conclusion}
\textbf{Thus , the distribution function is of Half-Normal type and the value of $k=2,\sigma=2$.}
\newpage
\section{References}
\vspace{30pt}
\begin{itemize}
    \item \url{https://docs.scipy.org}
    \item \url{http://geeksforgeeks.com}
    \item \url{https://wikipedia.org}
    \item \url{https://matplotlib.org/}
\end{itemize}
\end{document}